\begin{frame}
\frametitle{Better use of diagnostic information}
Information is lost when creating binary labels.\par
{\footnotesize
\begin{enumerate}
\item {\bf Normal subjects}: MMSE scores between 24-30 (inclusive), a CDR of 0, non-depressed, non MCI, and nondemented. The age range of normal subjects will be roughly matched to that of MCI and AD subjects. Therefore, there should be minimal enrollment of normals under the age of 70.
\item {\bf MCI subjects}: MMSE scores between 24-30 (inclusive), a memory complaint, have objective memory loss measured by education adjusted scores on Wechsler Memory Scale Logical Memory II, a CDR of 0.5, absence of significant levels of impairment in other cognitive domains, essentially preserved activities of daily living, and an absence of dementia.
\item {\bf Mild AD}: MMSE scores between 20-26 (inclusive), CDR of 0.5 or 1.0, and meets NINCDS/ADRDA criteria for probable AD.
\end{enumerate}
}
\vspace{0.25cm}
\begin{tiny}
\url{http://www.adni-info.org/scientists/ADNIGrant/ProtocolSummary.aspx}
\end{tiny}
\end{frame}

\begin{frame}
\frametitle{Uncertain labels}
\begin{quote}
...all clinically diagnosed AD patients will not have AD pathology, and up to 30\% of cognitively normal subjects will have a pathologic diagnosis of AD at autopsy.
\end{quote}

\begin{center}\begin{tiny}
Vemuri et al. ``Antemortem MRI based STructural Abnormality iNDex (STAND)-scores correlate with postmortem Braak neurofibrillary tangle stage''. NeuroImage 42(2):559--567 (2008).

\end{tiny}\end{center}
\end{frame}


