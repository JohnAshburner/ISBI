\begin{frame}
\frametitle{Model-based dimensionality reduction}
Nonlinear dimensionality reduction techniques related to PCA:
\begin{itemize}
\item {\bf Principal geodesic analysis} combins PCA with image registration to maximise the amount of signal explained.\par
\begin{tiny}
Zhang, Miaomiao, and P. Thomas Fletcher. ``Bayesian Principal Geodesic Analysis in Diffeomorphic Image Registration.'' Medical Image Computing and Computer-Assisted Intervention--MICCAI 2014. Springer International Publishing, 2014. 121-128.\par
\end{tiny}
\item {\bf Non-negative matrix factorization} decomposes data into matrices that are non-negative.\par
\begin{tiny}
Lee, Daniel D., and H. Sebastian Seung. ``Learning the parts of objects by non-negative matrix factorization.'' Nature 401.6755 (1999): 788-791.\par
Sotiras, Aristeidis, Susan M. Resnick, and Christos Davatzikos. ``Finding imaging patterns of structural covariance via Non-Negative Matrix Factorization.'' NeuroImage 108 (2015): 1-16.\par
\end{tiny}
\item {\bf Generalized principal components} incorporates a link function into principal component analysis.\par
\begin{tiny}
Collins, Michael, Sanjoy Dasgupta, and Robert E. Schapire. ``A generalization of principal components analysis to the exponential family.'' Advances in neural information processing systems. 2001.\par
Mohamed, Shakir, Zoubin Ghahramani, and Katherine A. Heller. ``Bayesian exponential family PCA.'' Advances in Neural Information Processing Systems. 2009.\par
\end{tiny}
\end{itemize}
\end{frame}

\begin{frame}
\frametitle{Principled similarity measures}
How do we best compute similarity from models used to ``pre-process'' data?
\begin{itemize}
\item {\bf Fisher kernels} T. S. Jaakkola and D. Haussler. ``Exploiting generative models in discriminative classifiers.'' In Kearns et al. [26], pages 487--493.
\end{itemize}
\end{frame}

\begin{frame}
\frametitle{Missing data}
\begin{columns}
\column{0.5\textwidth}
Brain images of different individuals have different fields of view.\par
If test subject has smaller field of view than training data, the pattern recognition approch needs to be retrained.\par
\column{0.5\textwidth}
\end{columns}
\end{frame}


