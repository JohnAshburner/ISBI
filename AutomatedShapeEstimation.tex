\begin{frame}
\frametitle{Non-Euclidean geometry}
\begin{columns}[c]
\column{0.5\textwidth}
\begin{itemize}
\item Distances are not always measured along a straight line.
\item ``\emph{Shapes are the ultimate non-linear sort of thing}''
\end{itemize}
\column{0.5\textwidth}
\includegraphics[width=\textwidth]{Globe}
\end{columns}
\end{frame}

\begin{frame}
\frametitle{Linear approximations to nonlinear problems}
%\begin{columns}[c]
%\column{0.2\textwidth}
%Dealing with non-Euclidean geometry.
%Linear approximation around the average.
%\begin{center}
%\column{0.8\textwidth}
\begin{center}
\includegraphics[width=1.2\textwidth]{spheres}
\end{center}
%\end{columns}
\end{frame}

%\begin{frame}
%\frametitle{D'Arcy Thompson's Generative Model}
%\begin{quote}
%``...diverse and dissimilar fishes can be referred as a whole to identical functions of very different co-ordinate systems...''
%\end{quote}
%\begin{center}
%\includegraphics[height=0.4\textheight]{OGAF}
%\includegraphics[height=0.4\textheight]{fish}
%\end{center}
%We can compute relative shapes using image registration.
%\end{frame}

