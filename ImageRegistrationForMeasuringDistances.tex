\begin{frame}
\frametitle{Image Registration}
\begin{columns}[c]
\column{0.7\textwidth}
\begin{itemize}
\item Image registration measures distances between images.
\item Often involves minimising the sum of two terms:
\begin{itemize}
\item Distance between the image intensities.
\item Distance of the deformation from the identity.
\end{itemize}
\item The sum of these terms gives the distance.
\end{itemize}
\column{0.3\textwidth}
\includegraphics[width=\textwidth]{shoot2d}
\end{columns}
\end{frame}

%%%%%%%%%%%%%%%%%%%%%%%%%%%%%%%%%%%%%%%%%%%%%%%%%%%%%%%%%%%%%%%
\begin{frame}
\frametitle{LDDMM}
\emph{Large Deformation Diffeomorphic Metric Mapping} is an image registration algorithm that minimises the following:
\begin{eqnarray*}
\mathcal{E}  =   \frac{1}{2} \int_{t=0}^1  || {\bf L} {\bf v}_t ||^2 dt +
                 \frac{1}{2\sigma^2} || f - \mu\left({\boldsymbol\varphi}_1^{-1}\right)||^2\\
\text{  where } {\boldsymbol\varphi}_0 = \mathrm{Id} \text{, } \frac{d{\boldsymbol\varphi}}{dt} = {\bf v}_t\left({\boldsymbol\varphi}_t\right)
\end{eqnarray*}

The first term minimises the squared distance measure of the deformations, whereas the second term simply minimises the difference between the warped template and the individual scan.

The objective is to estimate a series of velocity fields (${\bf v}_t$).\par
%\includegraphics[width=\textwidth]{trajectory}
%These may be conceptualised as
%\begin{eqnarray*}
%\frac{d {\boldsymbol\varphi}}{d t} = {\bf v}_t ({\boldsymbol\varphi})
%\end{eqnarray*}
\vspace{.25cm}
\begin{tiny}
Beg, MF, Miller, MI, Trouv{\'e}, A \& Younes, L. \emph{Computing large deformation metric mappings via geodesic flows of diffeomorphisms}. International Journal of Computer Vision 61(2):139--157 (2005).

\end{tiny}

\end{frame}

%%%%%%%%%%%%%%%%%%%%%%%%%%%%%%%%%%%%%%%%%%%%%%%%%%%%%%%%%%%%%%%

%\begin{frame}
%\frametitle{Different ways of measuring distances}
%\begin{columns}[c]
%\column{.2\textwidth}
%\begin{center}
%Two simulated images\par
%\includegraphics[width=\textwidth]{figure2Di}
%\end{center}
%\column{.8\textwidth}
%\includegraphics[width=\textwidth]{figure2Dii}
%\end{columns}
%\end{frame}



%\begin{frame}
%\frametitle{Change of Variables}
%When we warp images, we should usually account for expansion/contraction via a change of variables.
%\begin{eqnarray*}
%\int_{{\bf x} \in \varphi(\Omega)} f({\bf x}) d{\bf x} = \int_{{\bf x} \in \Omega} f(\varphi({\bf x})) \det |({\bf D}\varphi)({\bf x})| d{\bf x}
%\end{eqnarray*}
%where $({\bf D}\varphi)({\bf x})$ means the Jacobian of $\varphi$ at ${\bf x}$.
%\end{frame}

%%%%%%%%%%%%%%%%%%%%%%%%%%%%%%%%%%%%%%%%%%%%%%%%%%%%%%%%%%%%%%%

%\begin{frame}
%\frametitle{LDDMM}
%Estimating a series of velocity fields looks like it involves estimating a lot of parameters, but this is not actually the case.
%\includegraphics[width=\textwidth]{trajectory}

%The matching term of the objective function is:
%\begin{eqnarray*}
%\frac{1}{2\sigma^2} \int_{x\in\Omega} ( f \circ {\bf x} - \mu \circ {\boldsymbol\varphi}_{1}^{-1} \circ {\bf x})^2 d{\bf x}
%\end{eqnarray*}

%This may be re-written (including a change of variables) as:
%\begin{eqnarray*}
%\frac{1}{2\sigma^2} \int_{x\in\Omega} \det |{\bf D}( {\boldsymbol\varphi}_{1} \circ {\boldsymbol\varphi}_{t}^{-1}) \circ {\bf x} | ( f \circ {\boldsymbol\varphi}_1 \circ {\boldsymbol\varphi}_{t}^{-1} \circ {\bf x} - \mu \circ {\boldsymbol\varphi}_{t}^{-1} \circ {\bf x})^2 d{\bf x}
%\end{eqnarray*}
%This allows the derivatives of the matching term to be computed at any time point.
%\end{frame}

